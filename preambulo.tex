%
%				Preambulo feito do zero todo comentado para evitar a fadiga.
%
%				DATA: terça feira 01 de novembro de 2018
%
% Configuração da página
\usepackage{geometry}
\geometry{
	a4paper,
	vmargin={3cm,2cm},
	hmargin={3cm,2cm}
}
\usepackage[brazilian]{babel} 						%1º Idioma em Português
\usepackage[utf8]{inputenc} 						%2º Permite Acentos ç, é, ô ...
\usepackage[T1]{fontenc} 							% hifenização / Muitas fontes só funcionam com este complemento
%\usepackage{pythontex}  							% Python em LaTeX
%\usepackage{siunitx}	%Referente a PythonTex
%\usepackage{pythontex}	%Referente a PythonTex
\usepackage[thicklines]{cancel}						% Cancelar Frações APIN
\renewcommand{\CancelColor}{\color{red}}			% Muda a cor de cancel para red
\usepackage{xfrac}									% Este pacote fornece o comando \sfrac para criar frações inclinadas. 
\usepackage{lmodern} 								% Matematica Complemento do \sfrac
\usepackage{calligra} 								% Letra caligráfica
\usepackage{lipsum}									% Texto automático para praticar LaTeX
\usepackage{mathtools, amssymb, amsfonts, amsthm, latexsym, wasysym, siunitx} 	% Simbolos/Fonts Matemática
%no preâmbulo do documento. O pacote mathtools carrega o pacote amsmath e, portanto, não há necessidade de \usepackage{amsmath} no preâmbulo se o mathtools for usado.
\usepackage{bbold} 									% Escrita dos símbolos dos conjuntos (N,I,Z,Q,I)
\usepackage{graphicx} 								% Inclusão de imagens
\usepackage{framed}									% Destaca as resposta mais tem que ser utilizado com \definecolor{shadecolor}{gray}{.9}
\usepackage{wrapfig}								% Inclusao de imagens ao lado de texto...
\usepackage{float}									% Ajuda na hora de inserir imagens (Sobre a localização)"H"
\usepackage[final]{pdfpages} 						%Importar aquivos pdf \includepdf[pages=-]{arq.pdf}
\usepackage{tikz, tkz-base, tkz-fct} 				% Desenhos gráficos em LaTeX
\newcommand*\circled[1]{\tikz[baseline=(char.base)]{
		\node[shape=circle,draw,inner sep=2pt] (char) {#1};}}
\usepackage{verbatim} 								% Comentários em múltiplas linhas
\usepackage{multicol} 								% Texto em colunas
\usepackage{xcolor}									% Cores font ...
% [usenames, dvipsnames] Opicional					% Xcolor mais poderoso... 
\usepackage{pstricks} 								% Cores básicas {\red vermelho} {\blue azul}
\definecolor{minhacor}{RGB}{85,158,151}				% Cria uma cor personalizada RGB/rgb/html
%\usepackage{draftwatermark} %Marca dágua
%\usepackage{ifthen}									%Permite tomada de decisoes


\usepackage{fancybox}								% Responsavl pelas caixas de textos maravilhosas click Lembre-se: abrir documentação

%====================================================== FONTES ================================================================================================
% Begim Fontes
%\usepackage{fix-cm} % Permite aumentar o tamanho da fonte de fontes especíoficas além das padrões do LaTex
\usepackage{anyfontsize} 							% El tamaño de la fuente se puede controlar usando el paquete anyfontsize. podemos usar el comando \fontsize{t}{s}
			%\everymath{\displaystyle}	% Fontes normais em matemática sem encolher
\usepackage{wedn}
% End Fonts
%================================================================================================
\usepackage{qrcode}									% QR Code em Latex:
\usepackage{awesomebox}								% Alert/Advert/MsgBox
						%\notebox{Lorem ipsum...}\\
						%\tipbox{Lorem ipsum...}\\
						%\warningbox{Lorem ipsum...}\\
						%\cautionbox{Lorem ipsum...}\\
						%\importantbox{Lorem ipsum...}
\usepackage[shortlabels]{enumitem} 					% Versión 3.0 o + Personalizar el entorno “enumerate”.
\usepackage{polynom} 								% Resolucao de Polinomios
\usepackage{xlop}									% Deseha conta de dividir Ex: \opidiv{127}{7}\qquad\opdiv{127}{7}
\usepackage{venndiagram}							% Confeccao de Diagramas
\usepackage{booktabs} 								% Para tener opciones adicionales en el entorno tabular se puede utilizar el paquete booktabs
\usepackage{indentfirst}							% Ober identação da primeira linha de cada parágrafo é melhor deixar comentado \0/
\setlength{\parindent}{1,0cm}							% Deixa a identação em 0 (Impede a identação)
%\usepackage{parskip}								% Deixa um espaço entre parágrafo. EU COMENTEI O MESMO POIS O EXPASSO ESTAVA MUITO GRANDE
%\setlength{\parskip}{0,2cm} 							% Altera o parskip para 1cm - Distancia entre os parágrafos.
%\usepackage{setspace}								% Altera o espaçamento entre linhas
% https://aprendolatex.wordpress.com/tag/onehalfspacing/
												%	\onehalfspacing		% 1.5
													%\doublespacing		% Espacaento duplo
													%\setstretch{3}		% Espacamento personalizado
													%\singlespacing		% Espacamento simples que é o padrão
% Referencias - links
\usepackage[hyphens]{url}
\usepackage[breaklinks,colorlinks=true,linkcolor=red,citecolor=red, urlcolor=blue]{hyperref}
\usepackage{fancyvrb} 								% Notas de rodapé
\usepackage[normalem]{ulem} % Corrige o problema que ulem afeta em alguns estilos de bibliografia onde o texto em itálico é então sublinhado 
\usepackage{ulem} 									% Maneira de sublinhar/taxar um texto qualquer \sout{texto} ou \xout{texto}
\everymath{\displaystyle}							% Referente ao tamanho da fonte nos termos matemáticos ou em limites/derivadas

%============================== A BAIXO FICAM OS COMANDOS RENOMEADOS PARA AGILIZAR COMANDOS USADOS RECENTES====================
\newcommand{\R}{\mathbb{R}}								% Conjunto dos números Reais
\newcommand{\N}{\mathbb{N}}								% Conjunto dos números Naturais
\newcommand{\Z}{\mathbb{Z}}								% Conjunto dos números Inteiros
\newcommand{\Q}{\mathbb{Q}}								% Conjunto dos números Racionais
\newcommand{\PR}[1]{\ensuremath{\left[#1\right]}} 		%\PR que trabalha com Parênteses Rectos
\newcommand{\PC}[1]{$\scriptstyle#1${\left(#1\right)}}	%\PC para Parênteses Curvos e a terceira, para chavetas EXEMPLO: \PC{\frac{1}{2}}
\newcommand{\divpor}[2]{\opidiv{#1}{#2}}				% Traduzindo para facilitar...
\newcommand{\chav}[1]{\ensuremath{\left\{#1\right\}}}	%\chav adivinha... ;-)
\newcommand\vermelho[1]{\textcolor{red}{#1}}			% Vermelho
\newcommand\azul[1]{\textcolor{blue}{#1}}				% Azul
\newcommand\tab[1][1cm]{\hspace*{#1}}					% Tab funciona em ambiente matemático ou não.
\newcommand{\sen}{\mathop{\rm sen}\nolimits} 			% Seno
\newcommand{\Bascara}{x=\dfrac{-b\pm\sqrt{\Delta}}{2a}}	% Fórmula de Báscara
\newcommand{\raiz}{\sqrt}								% Traduzindo para facilitar...

% As 4 linhas abaixo representam aquela lista maravilhosa com efeito magnifico
\newcommand*{\itembolasazules}[1]{ % bolas 3D
	\footnotesize\protect\tikz[baseline=-3pt] %
	\protect\node[scale=.7, circle, shade, ball 
	color=blue]{\color{white}\Large\bf#1};}

\definecolor{shadecolor}{gray}{.9}			% Esta linha foi referenciada logo no começo do preambulo serve para destacar as respostas

%================== EURECA SEUS PROBLEMAS ACABARAM ======
\DeclareMathOperator{\implica}{\Rightarrow}
\DeclareMathOperator{\maiorigual}{\ge}
\DeclareMathOperator{\menorigual}{\le}
\DeclareMathOperator{\diferente}{\neq}
\DeclareMathOperator{\equivale}{\Leftrightarrow}
\DeclareMathOperator{\reticencia}{\dots}
\DeclareMathOperator{\vezes}{\times}
\DeclareMathOperator{\uniao}{\cup}
\DeclareMathOperator{\inter}{\cap}
\DeclareMathOperator{\paratodo}{\forall}
\DeclareMathOperator{\vazio}{\emptyset}


%============================== A CIMA FICAM OS COMANDOS RENOMEADOS ====================

\begin{comment}


71164170368 

beatriz 

NASCIMENTO  

ABDENEGO 


Outro parceiro

Dyones Santos da Costa.

[20:55, 25/1/2019] +55 98 9191-0679: CPF. 051.303.203-79
[20:56, 25/1/2019] +55 98 9191-0679: senha 34761433
 

SENHA: 

CPF PATRÃO 


%----------------------------------- mais dicas ----------------------------------
$
f(x)=
\left\{ 
    \begin{array}{rcl}
        x^2+1   & \mbox{se} & x\geq 0\\
                &           &\\
        \ln|x| & \mbox{si} & x< 0\\
    \end{array}
\right. 
$
apin http://xvideos.blog.br/sente-prazer-com-o-pau-do-namorado-na-boca/
%--------------------------------- modelo de sistema de equação-------------------
\end{comment}

% Configuração da página
\usepackage{geometry}
\geometry{
	a4paper,
	vmargin={2cm,2cm},
	hmargin={2cm,2cm}
}

%Minha Assinatura:
%\footnote{\LARGE{$\eta\varepsilon\tau\iota\eta  $}}\\


\begin{comment}
Observe a lista abaixo com alguns exemplos de prefixos:

ch:    capítulo
sec:  seção
subsec:  subseção
fig:  figura
tab:  tabela
eq:    equação
lst:  código fonte
itm:  lista de itens
alg:  algorítmo
app:  apêndice

Para nomearmos os objetos, utilizamos o comando \label{}

Agora, para referenciarmos algo, basta utilizarmos o comando \ref{}.


exemplos:
\section{Algoritmos padrão} \label{sec:algoritmos}

Voltemos à seção \ref{sec:algoritmos}, onde podemos observar...

Ainda temos mais um comando especial, que nos mostra o número da página onde encontra-se o elemento: \pageref{}.

Como podemos ver na figura \ref{img:proj-estacionamento}, que pode ser vista na página \pageref{img:proj-estacionamento}, o projeto final...


DICAS IMPORTANTES

% Site para baixar no Scribd em 04/09/19: https://www.scrdownloader.com/
% pacote que trata de questoes aleatórias  https://ctan.org/pkg/mcexam
% http://www.nead.ufma.br/portalava/?id=1|https://www.udemy.com/join/login-popup/?next=/home/my-courses/learning/|https://www.google.com.br/|https://github.com/Ferlinuxdebian/Debian-documentacao/blob/master/README.md|https://matrixcalc.org/pt/|https://pt.symbolab.com/|https://dlscrib.com/|https://github.com/EbookFoundation/free-programming-books/blob/master/free-programming-books-pt_BR.md#vim|https://www.tecmundo.com.br/iphone/2607-ios-como-criar-ringtones-no-itunes-10.htm%
% RM979848435CN
% RX868407165CN
% RL808060917CN
%	https://manas.tungare.name/software/isbn-to-bibtex	%Biblioteca automatica
%	http://www.codecogs.com/latex/eqneditor.php		%Editor de equacao para treinar
%	http://www.tug.dk/FontCatalogue/alphfonts.html \Catalogo de Fontes online em LaTeX
%	Site que ajuda na confecao de tabela em latex
%	https://www.tablesgenerator.com/#
%	https://www.latex-tables.com/
%	https://www.mathe-fa.de/pt % Fazer gráficos sem programas
%	Ajuda na fatoracoes
%	sudo mkfs.exfat -n hardDisk /dev/sdc2
%	O comando acima cria uma unidade exfat mais poderosa que o fat32 pois suporta arquivos acima de 4GB
\end{comment}
